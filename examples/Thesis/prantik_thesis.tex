% The UC Davis Dissertation Formatting Requirements can be found at
% http://www.gradstudies.ucdavis.edu/students/filing.html

% SPECIFY AN APPROPRIATE DOCUMENT CLASS WITH APPROPRIATE OPTIONS:
\documentclass[letterpaper, 12pt, oneside]{book} 
% for final version, uncomment above line and comment the line below
%\documentclass[letterpaper, 10pt, twoside,draft]{book} % draft


% SPECIFY THE PAGE AND HEADER FORMATTING (which is completely a hack):
    \usepackage[left=108pt, right=74pt, top=108pt, bottom=81pt, dvips, pdftex]{geometry}
    %\usepackage[thesis,draft,pdfinfo,dvips]%{ka-paper}

%new additions begin
		\usepackage{ae,aecompl}

		\usepackage{perpage}
		\MakePerPage{footnote}
		
		\usepackage{moreverb}
		%\usepackage[boxed]{algorithm2e}
		\usepackage[ruled,linesnumbered,noend]{algorithm2e}
				
		\usepackage{epsfig}
		\usepackage{graphicx}	% For including and formatting image files
		\usepackage{subfig}
		\usepackage{verbatim}	% for use of \begin{comment} \end{comment} statements
		\usepackage{multirow} % to use multirow in definition of tables

		\usepackage{float}
		\floatstyle{ruled}
		\newfloat{program}{thp}{lop}
		\floatname{program}{Program}
    
    \usepackage{fancyhdr} % See this package's documentation for more
    \pagestyle{fancy}     % information about the following commands
    \renewcommand{\sectionmark}[1]{\markright{\thesection.\ #1}}
    \fancyhf{}
    \fancyhead[R]{\thepage}
    %\fancyhead[L]{\rightmark}
    \renewcommand{\headrulewidth}{0pt} % change this to put a line between
                                       % the header and the main text
    
    % The following command sets the name of the Bibliography:
    \renewcommand{\bibname}{Bibliography}

    
    % EXTREMELY COMMON LaTeX PACKAGES TO INCLUDE:
    \usepackage{amsmath,amsthm, amsfonts,amssymb} % For AMS Beautification
    \usepackage{setspace} % For Single & Double Spacing Commands
    \usepackage[linktocpage,bookmarksopen,bookmarksnumbered,% For PDF navigation
                pdftitle={PhD Dissertation by Prantik Bhattacharyya},%   and URL hyperlinks
                pdfauthor={Prantik Bhattacharyya},%
                pdfsubject={PhD Dissertation},%
                pdfkeywords={UC Davis PhD Dissertation, Prantik Bhattacharyya}]{hyperref}
    
    % SOME SLIGHTLY LESS COMMON LaTeX PACKAGES TO INCLUDE 
    % (uncomment as necessary)
%     \usepackage{pstricks, pst-plot} % For creating images & figures
%     \usepackage[vcentermath]{youngtab} % For typesetting Young Tableaux

    % COMPLETELY OPTIONAL LaTeX PACKAGES TO INCLUDE:
    % (uncomment as necessary)
%     \usepackage{lineno} % For Line Numbering
%     \usepackage{appendix} % For Control of Appendix Numbering & Location

   
    
    % DEFINE SOME USEFUL THEOREMS, ENVIRONMENTS, ETC.:
 		\newtheorem{definition}{Definition}
		\newtheorem{theorem}{Theorem}

    \newtheorem{Theorem}{Theorem}[section]
    \newtheorem{Proposition}[Theorem]{Proposition}
    \newtheorem{Lemma}[Theorem]{Lemma}
    \newtheorem{Corollary}[Theorem]{Corollary}
    \theoremstyle{definition}
        \newtheorem{Definition}[Theorem]{Definition}
        \newtheorem{Example}[Theorem]{Example}
        \newtheorem{Conjecture}[Theorem]{Conjecture}
        \newtheorem{Problem}[Theorem]{Problem}
        \newtheorem{Algorithm}[Theorem]{Algorithm}
        \newtheorem{CardGame}[Theorem]{Card Game}
        \newtheorem{Strategy}[Theorem]{Strategy}
        \newtheorem{Question}[Theorem]{Question}
    \theoremstyle{remark}
        \newtheorem{Remark}[Theorem]{Remark}
    \newenvironment{Sketch}{\par\noindent{\sc Sketch of Proof}\quad}{\hfill\qed\par\smallskip}
    \newenvironment{Proof}{\par\noindent{\sc Proof}\quad}{\hfill\qed\par\smallskip}
    \newenvironment{ReuseTheorem}[2]{\par\vspace{12pt}\noindent{\bf #1~\ref{#2}$'$.}\it}{\par\vspace{12pt}}
    \newenvironment{RestateTheorem}[2]{\par\vspace{12pt}\noindent{\bf #1~\ref{#2}.}\it}{\par\vspace{12pt}}


		\newcommand{\si}{\begin{enumerate}} 
		\newcommand{\ii}{\item} 
		\newcommand{\ei}{\end{enumerate}} 
		
		\def \nvspace {0in}

    % DETERMINE HOW EXTENSIVELY TO NUMBER EQUATIONS AND FIGURES:
    \numberwithin{equation}{section}
    \numberwithin{figure}{section}
    
    % These commands can be placed in the preamble (to be a global definition)
% or locally inside the \begin{table}...\end{table} syntax to adjust vertical space inside table cells.
\newcommand\T{\rule{0pt}{2.6ex}}
\newcommand\B{\rule[-1.2ex]{0pt}{0pt}}
    
    % DEFINE NEW ANY COMMANDS (for your own convenience):
    \newcommand{\n}{\vspace{12pt}} % for typesetting line breaks with 12
                                   % Postscript points of extra whitespace
      
    \newcommand{\newchapter}[3] % for typesetting new chapters with the
	{                           % correct initial page numbering style
        % Arguments: (#1) Short name for chapter, which is used in 
        %                 any running headers
        %            (#2) Medium length name for chapter, which is
        %                 used in the table of contents
        %            (#3) Long name for chapter, which is typeset at 
        %                 the starting the chapter
        \chapter[#2]{#3}
        \chaptermark{#1}
        \thispagestyle{myheadings}
	}


\begin{document}
    
    % Toggle commenting the following command out in order to toggle 
    % the inclusion of line numbering (this requires the ``lineno'' 
    % package to be includes above):
%     \linenumbers
    

    % The following commands produce page numbering at the bottom
    % center using roman numerals per UC Davis requirements for the
    % front matter of the dissertation:
    \pagenumbering{roman}
    \pagestyle{plain}
    
       
    % %------------------------------------------------------------------------------
    % 
    % %---------------------------- NEW PAGE ----------------------------------------
    %
    % %---------------------------- TITLE PAGE---------------------------------------


    % The following command produces single-spaced lines for the title page:
    \singlespacing

    ~\vspace{-1in} % to force the title up a bit higher
    \begin{center}

       \begin{huge}
            Thesis Title
        \end{huge}
        \\\n\n
        By\\\n
        {\sc PRANTIK BHATTACHARYYA}\\\n
        M.S. (University of California, Davis) 2009\n\n      
         
%        A technical Reported submitted for\\\n
%        Ph.D. Qualifying Examination\\\n
%				in \\\n
%        COMPUTER SCIENCE\\\n
%				in the\\\n
%        OFFICE OF GRADUATE STUDIES\\\n
%        of the\\\n
%        UNIVERSITY OF CALIFORNIA\\\n
%        DAVIS\\\n\n

        DISSERTATION\\\n
        Submitted in partial satisfaction of the requirements for the degree of\\\n
        DOCTOR OF PHILOSOPHY\\\n
        in\\\n
        COMPUTER SCIENCE\\\n
        in the\\\n
        OFFICE OF GRADUATE STUDIES\\\n
        of the\\\n
        UNIVERSITY OF CALIFORNIA\\\n
        DAVIS\\\n
%        
        Approved:\\\n\n
          
        \rule{4in}{1pt}\\
        ~Committee Chair\\\n\n
        
        \rule{4in}{1pt}\\
        ~Committee Member\\\n\n
        
       \rule{4in}{1pt}\\
        ~Committee Member\\\n

      %  \vfill
        
        Committee in Charge\\
        ~2013

    \end{center}

    \newpage
    
       
    % %------------------------------------------------------------------------------
    % 
    % %------------------------------- NEW PAGE -------------------------------------
    %
    % %------------------------------ COPYRIGHT PAGE --------------------------------
        
    % The following commands create a copyright notice page. This 
    % page can be deleted if you would prefer to not include it.
        

    ~\\[7.75in] % to place the copyright near the bottom of the page
    \centerline{Copyright \copyright\ Prantik Bhattacharyya, 2013.}
    \centerline{All rights reserved.}

   \thispagestyle{empty}
    \addtocounter{page}{-1}

    \newpage
    
    % %------------------------------------------------------------------------------      
    % %------------------------------------------------------------------------------
    % 
    % %---------------------------------- NEW PAGE ----------------------------------
    %
    % %------------------------------ DEDICATION PAGE -------------------------------
    
    % The following command inserts a nice dedication to whomever you
    % feel is appropriate.  This page can be deleted if you would
    % prefer to not include it.
	    \centerline{To friends and family.}
	    
	    \newpage


    % The following command produces double-spaced lines for the
    % remainder of the document:
%    \doublespacing
    
    % 
    % %---------------------------------- NEW PAGE ----------------------------------
    %
    % %---------------------------------- ABSTRACT ----------------------------------
    
    % Toggle the commenting of the following command in order to
    % include a page number.  The purpose of this is to allow the
    % create an autonomous Abstract Page per University Requirements.
    % 
    % Note that the page numbering style for the autonomous Abstract 
    % Page needs to be adjusted. See the Grad Studies website.
    %\thispagestyle{empty}

    ~\vspace{-1in} % to force the attribution text up a bit higher
    \begin{flushright}
        \singlespacing
        Prantik Bhattacharyya\\
        Computer Science\\
        University of California, Davis\\\n
    \end{flushright}

    \centerline{\large Thesis Title}
    
    \vspace{0.3in}
    
    \centerline{\textbf{\underline{Abstract}}}
    
    \vspace{0.2in}
    
        % Either type your Abstract Text here or input a file 
        % containing it:
        \input{texfiles/abstract}
    
    \newpage
    
    % %------------------------------------------------------------------------------
    % 
    % %---------------------------------- NEW PAGE ----------------------------------
    %
    % %------------------------------ Keywords -------------------------------

    \chapter*{\vspace{5in}}
    % the vspace command forces the title up a bit higher 

        % Either type your Keywords Text here or input a file 
        % containing it:
        \textbf{Keywords:} UC Davis PhD Thesis, Latex Templates.

    
    \newpage


       
    % %------------------------------------------------------------------------------
    % 
    % %---------------------------------- NEW PAGE ----------------------------------
    %
    % %------------------------------ ACKNOWLEDGMENTS -------------------------------

    \chapter*{\vspace{-1.5in}Acknowledgments}
    % the vspace command forces the title up a bit higher 

        % Either type your Acknowledgments Text here or input a file 
        % containing it:
        \input{texfiles/acknowledgments}
    
    \newpage
       
    % %------------------------------------------------------------------------------
    % 
    % %---------------------------------- NEW PAGE ----------------------------------
    %
    % %------------------------------ TABLE OF CONTENTS -----------------------------
   
    
    % The following command creates the table of contents:
    \tableofcontents
    %\setcounter{tocdepth}{2}
    
    % The following command, though it follows the above 
    % ``\tableofcontents'' command, actually causes the Table of 
    % Contents to start on a new page:
    \newpage
    
    \listoffigures 
    \listoftables 
    \newpage
     
    \addcontentsline{toc}{section}{List of Figures}
    \addcontentsline{toc}{section}{List of Tables}
    

    % The following command produces double-spaced lines for the
    % remainder of the document:
    \doublespacing
    
    
    
    % %------------------------------------------------------------------------------
    % 
    % %---------------------------------- NEW PAGE ----------------------------------
    %
    % %---------------------------- BEGIN TEXT OF THESIS ----------------------------



    % The following commands produce page numbering at the top right
    % using arabic numerals per UC Davis requirements for the main
    % text of the dissertation:
    \pagestyle{fancy}
    \pagenumbering{arabic}
       
    % %------------------------------------------------------------------------------
    % 
    % %--------------------------------- NEW CHAPTER --------------------------------
    %
    % %-------------------------------- INTRODUCTION --------------------------------

    \newchapter{Introduction}{Introduction}{Introduction}
    \label{sec:Intro}


        % Either type your Introduction here or input a file 
        % containing it using the ``\input'' command.
        
        % You will probably want to split your chapter up into several
        % sections (with each section possibly even split up into
        % subsections), each of which can either be written directly
        % in this file or input from an external file as above.
        %
        % Suggested sections for your introduction include a general 
        % overview (with historical motivation as appropriate) and a 
        % summary of your main results. E.g.:
    
    
        \label{sec:Intro:Introduction}

This is the introductory section.
        
        \section{Research Contributions}
        \label{sec:Intro:research_contribution}
        Discuss your research contribution here.        
        
        
    % %------------------------------------------------------------------------------
    % 
    % %--------------------------------- NEW CHAPTER --------------------------------
    %
    % %----------------------- [INSERT TITLE OF CHAPTER TWO] ------------------------
    
    
    \newchapter{Chapter One Title}{Chapter One Title}{Chapter One Title}
    \label{sec:chapterOne}
	\section{Introduction}
\label{sec:chapterOne:intro}
Chapter One Introduction. This template can be cited as \cite{latex_template}.
    

        % Either type your chapter's text here or input a file 
        % containing it using the ``\input'' command.
        
        % You will probably want to split your chapter up into several
        % sections (with each section possibly even split up into
        % subsections), each of which can either be written directly
        % in this file or input from an external file as above. E.g.:
		

        % You should repeat this portion for each additional chapter 
        % as it's appropriate to your dissertation.
                  
        
    % %------------------------------------------------------------------------------
    % 
    % %-------------------------------- NEW PAGE ------------------------------------
    %
    % %------------------------------- REFERENCES ----------------------------------
    
    % remember to change bibliography settings for final version
    
        %\input{references}
      
      
\bibliographystyle{abbrv}
\bibliography{references}

\addcontentsline{toc}{chapter}{Bibliography}
\end{document}
