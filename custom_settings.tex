\usepackage{ifpdf}
\usepackage{amsmath,  amssymb}

\usepackage{mathptmx}       % selects Times Roman as basic font
\usepackage{helvet}         % selects Helvetica as sans-serif font
\usepackage{courier}        % selects Courier as typewriter font
\usepackage{type1cm}        % activate if the above 3 fonts are
                            % not available on your system
                            
\usepackage{makeidx}         % allows index generation
\usepackage{multicol}        % used for the two-column index
\usepackage[bottom]{footmisc}% places footnotes at page bottom


\usepackage{graphicx}
  %\usepackage[pdftex]{graphicx} %don't use [pdftex] attribute to support import of graphics object belonging to eps,ps categories
  
  % declare the path(s) where your graphic files are
  % and their extensions so you won't have to specify these with
  % every instance of \includegraphics
   \ifpdf
   	\DeclareGraphicsExtensions{.pdf,.jpeg,.png}
   	\graphicspath{{images/}{images/pdf/}}
   \else
   	\DeclareGraphicsExtensions{.ps,.eps}
   	\graphicspath{{images/}{images/ps/}}
   \fi
   

\usepackage{float}
\usepackage{subfig}
\usepackage{verbatim}	% for use of \begin{comment} \end{comment} statements
\usepackage{multirow} % to use multirow in definition of tables

\newcommand{\si}{\begin{enumerate}}
\newcommand{\ii}{\item}
\newcommand{\ei}{\end{enumerate}}

% These commands can be placed in the preamble (to be a global definition)
% or locally inside the \begin{table}...\end{table} syntax to adjust vertical space inside table cells.
\newcommand\T{\rule{0pt}{2.6ex}}
\newcommand\B{\rule[-1.2ex]{0pt}{0pt}}

\makeindex     